\documentclass[12pt]{scrbook}

%%%%%%%%%%%%%%%%%%%%%%%%%%%%%%%%%%%%%%%%%%%%%%%%%%%%%%%%%%%%%%
% Useful packages
\usepackage{mathtools}
\usepackage{amssymb,bm,bbold}
\usepackage{enumerate}

\usepackage{hhline}
\usepackage{float}

% CSCI-265
\usepackage{tikz}
\usetikzlibrary{automata, positioning, arrows, arrows.meta}


%=================================
% pre-defined theorem environments
\usepackage{amsthm}
\newtheorem{theorem}{Theorem}
\newtheorem{lemma}{Lemma}
\newtheorem{proposition}{Proposition}
\newtheorem{corollary}{Corollary}
\newtheorem{definition}{Definition}
\newtheorem*{remark}{Remark}
\newtheorem*{assumption}{Assumption}

%=================================
% useful commands
\DeclareMathOperator*{\argmin}{arg\,min}
\DeclareMathOperator*{\argmax}{arg\,max}
\DeclareMathOperator*{\supp}{supp}

\def\vec#1{{\ensuremath{\bm{{#1}}}}}
\def\mat#1{\vec{#1}}

%=================================
% convenient notations
\newcommand{\XX}{\mathbb{X}}
\newcommand{\RR}{\mathbb{R}}
\newcommand{\NN}{\mathbb{N}}
\newcommand{\QQ}{\mathbb{Q}}
\newcommand{\ZZ}{\mathbb{Z}}
\newcommand{\EE}{\mathbb{E}}
\newcommand{\PP}{\mathbb{P}}

\newcommand{\sL}{\mathcal{L}}
\newcommand{\sX}{\mathcal{X}}
\newcommand{\sY}{\mathcal{Y}}

\newcommand{\ind}{\mathbb{1}}

\newcommand{\kleene}{{}^\ast}

%%%%%%%%%%%%%%%%%%%%%%%%%%%%%%%%%%%%%%%%%%%%%%%%%%%%%%%%%%%%%%%%%%%%%
% Typography, change document font
\usepackage[tt=false, type1=true]{libertine}
\usepackage[varqu]{zi4}
\usepackage[libertine]{newtxmath}
\usepackage[T1]{fontenc}
%\usepackage[margin=2.5cm]{geometry} % margins
%\usepackage[english]{babel} % language (replace with german or ngerman for german texts)
%\usepackage[utf8]{inputenc} % Umlaute
%\usepackage{amsmath}    % math formulas
%\usepackage{graphicx}    % graphics
%\usepackage{fancyhdr}    % header and footer on every page
%\usepackage{setspace}    % line space (e.g. \singlespacing, \onehalfspacing or \doublespacing)
%\usepackage{enumitem}     % enumerate options
\usepackage[protrusion=true,expansion=true]{microtype}

\author{Guy Matz}

\setlength{\parindent}{0pt} % first line in paragraph will not be indented
%\onehalfspacing        % 1.5 line spacing

\begin{document}

%% Header on first page with course information etc. %%
  \begin{tabular}{p{15.5cm}}
    {\large \textbf{Computer Theory I}} \\
    Hunter College \\
    Spring 2023  \\
    Instructor: Prof. Tojeira\\
    \hline
  \end{tabular}
  \begin{center}
    {\Large \textbf{Problem Set 11}}
    \vspace{2mm}\\
    Guy Matz
  \end{center}

  Definitions
  \begin{itemize}
    \item Recursively Enumerable: A Turing Machine accepts the language
    \item Recursive: A Language is recursive is some Turing Machine accepts
      it and it halts on any input.
    \item Membership: Being able to definitively tell what is in a language
      and what is not
    \item Context-Sensitive: A language is Context-sensitive if some
      non-deterministic Turing Machine accepts it, using linear space
    \item Linear Space: Space limited to a linear function of the size of input
    \item
  \end{itemize}

\begin{enumerate}
  \item \textbf{Proving the relations of the Chomsky hierarchy}.  Give brief
    justifications for each of the following relations
    \begin{enumerate}
      \item The class of CFLs is a proper subset of the class of CSLs.

        Every RE is Recursive, but there are some Recursive languages that are
        not Recursively Enumerable.

      \item The class of recursively enumerable languages is a superset of class of recursive languages.

        Every RE is Recursive, but there are some Recursive languages that are
        not Recursively Enumerable.

      \item The class of CSLs is a subset of the class of recursive languages.

        A Turing machine can process both a RE language as well as a CSL, while
        Context Sensitive language can be processed by a linear-bounded automaton

    \end{enumerate}
  \item \textbf{Turing Machine Algorithms}.  Describe the approach you might take with a
    Turing Machine to accept the following languages. You do not need to
    write out the TM, just describe how it would operate.

      \begin{enumerate}
        \item All palindromes of even length

          \begin{enumerate}
            \item Mark the first letter as \#
              \begin{enumerate}
                \item If it was an $a$, go R until $B$, then go L past \#s
                  and look for an $a$
                  \begin{itemize}
                    \item If not an $a$, FAIL
                  \end{itemize}
                \item If it was an $b$, go R until $B$, then go L past \#s
                  and look for an $b$
                  \begin{itemize}
                    \item If not $b$, FAIL
                  \end{itemize}
              \end{enumerate}
            \item Go L until $B$
            \item Skip \#, go R
            \item Repeat
          \end{enumerate}

        \item $a^{2^n}$, $n \geq 0$ (words with a number of a's equal to some power of 2)

          \begin{enumerate}
            \item Crash on any letter other than $a, \#, B$
            \item In $q_0$
              \begin{itemize}
                \item $B$: HALT
                \item \#: Go R
                \item $a$ : Mark as \# and enter $q_1$
              \end{itemize}
            \item In $q_1$
              \begin{itemize}
              \item $a$: go R
              \item \#: go R
              \item $B$: Enter $q_2$, go L
              \end{itemize}
            \item In $q_2$
              \begin{itemize}
              \item $a$: Mark as \# and enter $q_3$
              \item \#: Go L
              \end{itemize}
            \item In $q_3$
              \begin{itemize}
              \item $a$: Go L
              \item \#: Go L
              \item $B$: Enter $q_0$ and repeat
              \end{itemize}
          \end{enumerate}

          \begin{center}
          \begin{tabular}{| c || c | c | c | c |}
            \hline
              & q_0& q_1 & _2 & q_3   \\
            \hline \hline
            a & q_1, \#, R & q_1, a, R & q_3, \#, L & q_3, a, L   \\
            \hline
            \# & q_0, \#, R & q_1, \#, R &  q_2, \#, L  & q_2, \#, L  \\
            \hline
            B & \text{HALT} &  q_2, B, L &  & q_0, B, L \\
            \hline
          \end{tabular}
          \end{center}

        \item $a^p$, where p is prime. Hint: you can add additional letters or symbols as needed.

          I don't know :-(

      \end{enumerate}

  \item \textbf{Closure Properties}.  Prove or disprove each of the following closure properties
    \begin{enumerate}
          \item CSLs are closed under complement

            False.  CSLs include CFLs, which are not closed under complement
            so CSLs are not either

          \item CSLs are closed under intersection

            False.  Since CSLs are not closed under complement, by
            DeMorgan's, they are not closed under intersection

          \item Recursive languages are closed under concatenation

            True.  Recursive languages are accepted by a Turing Machine,
            so - by construction - concatentating languages are accepted
            by a TM

          \item The subset of a recursive language is always recursive

            True.  Recursive languages are accepted by a Turing Machine,
            so a subset off the language will also be accepted
            by a TM

          \item RE (recursively enumerable) languages are closed under complement

            False.  I don't know why, but it has to be.


    \end{enumerate}

\end{enumerate}
\end{document}
