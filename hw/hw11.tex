\documentclass[12pt]{scrbook}

%%%%%%%%%%%%%%%%%%%%%%%%%%%%%%%%%%%%%%%%%%%%%%%%%%%%%%%%%%%%%%
% Useful packages
\usepackage{mathtools}
\usepackage{amssymb,bm,bbold}
\usepackage{enumerate}

\usepackage{hhline}
\usepackage{float}

% CSCI-265
\usepackage{tikz}
\usetikzlibrary{automata, positioning, arrows, arrows.meta}


%=================================
% pre-defined theorem environments
\usepackage{amsthm}
\newtheorem{theorem}{Theorem}
\newtheorem{lemma}{Lemma}
\newtheorem{proposition}{Proposition}
\newtheorem{corollary}{Corollary}
\newtheorem{definition}{Definition}
\newtheorem*{remark}{Remark}
\newtheorem*{assumption}{Assumption}

%=================================
% useful commands
\DeclareMathOperator*{\argmin}{arg\,min}
\DeclareMathOperator*{\argmax}{arg\,max}
\DeclareMathOperator*{\supp}{supp}

\def\vec#1{{\ensuremath{\bm{{#1}}}}}
\def\mat#1{\vec{#1}}

%=================================
% convenient notations
\newcommand{\XX}{\mathbb{X}}
\newcommand{\RR}{\mathbb{R}}
\newcommand{\NN}{\mathbb{N}}
\newcommand{\QQ}{\mathbb{Q}}
\newcommand{\ZZ}{\mathbb{Z}}
\newcommand{\EE}{\mathbb{E}}
\newcommand{\PP}{\mathbb{P}}

\newcommand{\sL}{\mathcal{L}}
\newcommand{\sX}{\mathcal{X}}
\newcommand{\sY}{\mathcal{Y}}

\newcommand{\ind}{\mathbb{1}}

\newcommand{\kleene}{{}^\ast}

%%%%%%%%%%%%%%%%%%%%%%%%%%%%%%%%%%%%%%%%%%%%%%%%%%%%%%%%%%%%%%%%%%%%%
% Typography, change document font
\usepackage[tt=false, type1=true]{libertine}
\usepackage[varqu]{zi4}
\usepackage[libertine]{newtxmath}
\usepackage[T1]{fontenc}
%\usepackage[margin=2.5cm]{geometry} % margins
%\usepackage[english]{babel} % language (replace with german or ngerman for german texts)
%\usepackage[utf8]{inputenc} % Umlaute
%\usepackage{amsmath}    % math formulas
%\usepackage{graphicx}    % graphics
%\usepackage{fancyhdr}    % header and footer on every page
%\usepackage{setspace}    % line space (e.g. \singlespacing, \onehalfspacing or \doublespacing)
%\usepackage{enumitem}     % enumerate options
\usepackage[protrusion=true,expansion=true]{microtype}

\author{Guy Matz}

\setlength{\parindent}{0pt} % first line in paragraph will not be indented
%\onehalfspacing        % 1.5 line spacing

\begin{document}

%% Header on first page with course information etc. %%
  \begin{tabular}{p{15.5cm}}
    {\large \textbf{Computer Theory I}} \\
    Hunter College \\
    Spring 2023  \\
    Instructor: Prof. Tojeira\\
    \hline
  \end{tabular}
  \begin{center}
    {\Large \textbf{Problem Set 11}}
    \vspace{2mm}\\
    Guy Matz
  \end{center}

  Definitions
  \begin{itemize}
    \item Recursively Enumerable: A Turing Machine accepts the language
      \begin{itemize}
        \item Generated by Unrestricted Grammars
        \item Semi-Decidable
        \item Recognizable: Accepts strings in the language, may not
          terminate if it can't tell
      \end{itemize}
      \item Recursive: A Language is recursive if it can be recognized by a
        Turing Machine that is guaranteed to halt (Decidability)
      \begin{itemize}
        \item A subset of Recursively Enumerable
        \item Decidable: Accept words in $L$, reject $w \notin L$
      \end{itemize}
    \begin{itemize}
      \item Context-Sensitive: A language that can be recognized by a Non-Deterministic Turin Machine
        using linear space
      \begin{itemize}
        \item Recognized by LBA
        \item Example: $a^nb^na^n$
      \end{itemize}
      \item P: Class of all "Decision Problems" that can be solved in Polynomial time
        on a Deterministic Turing machine.  Is a subset of Context-Sensitive lang
      \item NP: Class of all "Decision Problems" that can be solved in Polynomial time
        on a NON-Deterministic Turing machine.
    \end{itemize}
    \item Membership: Being able to definitively tell what is in a language
      and what is not
    \item Linear Space: Space limited to a linear function of the size of input

  \end{itemize}

\begin{enumerate}
  \item \textbf{Proving the relations of the Chomsky hierarchy}.  Give brief
    justifications for each of the following relations
    \begin{enumerate}
      \item The class of CFLs is a proper subset of the class of CSLs.

        Every CFL can be processed by a Linearly Bounded Automata,
        but there are CSLs that cannot be processed by a PDA

        or

        A CFG is a CSG whose production head is limited to containing exactly
        one variable, so it is therefore a restricted CSG, and is therefore a
        proper subset.

        Prof says:  Show a word is in the superset, but not in the subset.  $a^nb^nc^n$ is 
        CSL but not CFL (see context-sensitive tasks pdf).  And an LBA can simulate a PDA

        or

        a CFG is a restricted form of a CSG

      \item The class of recursively enumerable languages is a superset of class of recursive languages.

        Every RE is Recursive, but there are some Recursive languages that are
        not Recursively Enumerable.

        Prof says: A RE is a lang for whcih there is a TM that will recognize, accept and
        halt for all words in the languages.  A recursive lang will also reject
        words not in the language

      \item The class of CSLs is a subset of the class of recursive languages.

        A Turing machine can process both a RE language as well as a CSL, while
        Context Sensitive language can be processed by a linear-bounded automaton

        Prof says: CSGs can be transformed into Kuroda normal form, but recursive languages
        not in CSG cannot

        OR a LBA can be simulated by a TM using a linear amount of tape.
        OR an LBA is a restricted TM

    \end{enumerate}
  \item \textbf{Turing Machine Algorithms}.  Describe the approach you might take with a
    Turing Machine to accept the following languages. You do not need to
    write out the TM, just describe how it would operate.

      \begin{enumerate}
        \item All palindromes of even length

          \begin{enumerate}
            \item Mark the first letter as \#
              \begin{enumerate}
                \item If it was an $a$, go R until $B$, then go L past \#s
                  and look for an $a$
                  \begin{itemize}
                    \item If not an $a$, FAIL
                  \end{itemize}
                \item If it was an $b$, go R until $B$, then go L past \#s
                  and look for an $b$
                  \begin{itemize}
                    \item If not $b$, FAIL
                  \end{itemize}
              \end{enumerate}
            \item Go L until $B$
            \item Skip \#, go R
            \item Repeat
          \end{enumerate}

        \item $a^{2^n}$, $n \geq 0$ (words with a number of a's equal to some power of 2)

          \begin{enumerate}
            \item Crash on any letter other than $a, \#, B$
            \item In $q_0$
              \begin{itemize}
                \item $B$: HALT
                \item \#: Go R
                \item $a$ : Mark as \# and enter $q_1$
              \end{itemize}
            \item In $q_1$
              \begin{itemize}
              \item $a$: go R
              \item \#: go R
              \item $B$: Enter $q_2$, go L
              \end{itemize}
            \item In $q_2$
              \begin{itemize}
              \item $a$: Mark as \# and enter $q_3$
              \item \#: Go L
              \end{itemize}
            \item In $q_3$
              \begin{itemize}
              \item $a$: Go L
              \item \#: Go L
              \item $B$: Enter $q_0$ and repeat
              \end{itemize}
          \end{enumerate}

          \begin{center}
          \begin{tabular}{| c || c | c | c | c |}
            \hline
              & q_0& q_1 & _2 & q_3   \\
            \hline \hline
            a & q_1, \#, R & q_1, a, R & q_3, \#, L & q_3, a, L   \\
            \hline
            \# & q_0, \#, R & q_1, \#, R &  q_2, \#, L  & q_2, \#, L  \\
            \hline
            B & \text{HALT} &  q_2, B, L &  & q_0, B, L \\
            \hline
          \end{tabular}
          \end{center}

        \item $a^p$, where p is prime. Hint: you can add additional letters or symbols as needed.

          I don't know :-(

          Prof says: See example of divisibility.  Add more $b$s to check ffor primeness

      \end{enumerate}

  \item \textbf{Closure Properties}.  Prove or disprove each of the following closure properties
    \begin{enumerate}
          \item CSLs are closed under complement

            False.  CSLs include CFLs, which are not closed under complement
            so CSLs are not either

            Prof says: Yes.!  Swap Halt & Crash states

          \item CSLs are closed under intersection

            False.  Since CSLs are not closed under complement, by
            DeMorgan's, they are not closed under intersection

            Prof says: Since CSLs are closed under complement, by
            DeMorgan's, they are closed under intersection (Union by construction)

          \item Recursive languages are closed under concatenation

            True.  Recursive languages are accepted by a Turing Machine,
            so, by construction, concatentating languages are accepted
            by a TM

            Prof says: Apply construction $S -> XY$ (aaplied to union & kleene)

          \item The subset of a recursive language is always recursive

            True.  Recursive languages are accepted by a Turing Machine,
            so a subset of the language will also be accepted
            by a TM

            Prof says: False.  $\(a + b\)\kleene$ is outside of recursive
            (but not ffinitely describable)

          \item RE (recursively enumerable) languages are closed under complement

            False.  I don't know why, but it has to be.

            Prof says: For REs, we sometimes have undecidable words,
            so we can't tell if those words are in the language ?


    \end{enumerate}

    4.a individual Turing machines and individual C++ programs are finite.
    Diagonilzation proofs only work on infinite sets

    

\end{enumerate}
\end{document}
