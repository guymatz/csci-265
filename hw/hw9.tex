\documentclass[11pt, a4paper]{article} % setsfont size and layout

%% required packages %%

\usepackage[margin=2.5cm]{geometry} % margins
\usepackage[english]{babel} % language (replace with german or ngerman for german texts)
\usepackage[utf8]{inputenc} % Umlaute
\usepackage{amsmath}		% math formulas
\usepackage{graphicx}		% graphics
\usepackage{fancyhdr}		% header and footer on every page
\usepackage{setspace}		% line space (e.g. \singlespacing, \onehalfspacing or \doublespacing)


%% Here the main part of the document begins %%

\begin{document}

%% Some more settings %%

\setlength{\parindent}{0pt} % first line in paragraph will not be indented
\onehalfspacing				% 1.5 line spacing
%\thispagestyle{empty}		% no header and page number on first page


%% Header on first page with course information etc. %%
\begin{tabular}{p{15.5cm}}
      {\large \textbf{Computer Theory I}} \\
	Hunter College \\
    Spring 2023  \\
	Instructor: Prof. Tojeira\\
	\hline
	\\
\end{tabular}

\vspace*{0.3cm}				% vertical space between header on top of the page and main heading


\begin{center}
	{\Large \textbf{Problem Set 8}}
	\vspace{2mm}

	Guy Matz

\end{center}

\vspace{0.4cm}
\section*{Problem 1}
\begin{enumerate}
  \item \textbf{Create an unambiguous grammar which generates basic
    mathematical expressions, using numbers and the four operators: +, -, *, /.
  Abstract "number" as a single terminal, n, so that words in the language may
  look like: n+n*n, etc.}
  \\
\emph{Parsing and mathematically evaluating expressions created by this string
  should give the result you'd get while following the usual order of
  operations. In the order of operations, * and / should be given precendence
  over + and -. For operators of equivalent precedence, evaluation should occur
  from left to right.\\
So 8/4*2 would result in 4, not 1.\\
1+2*3 would be 7.\\
4+3*5-6/2*4 would be 4+15-6/2*4 = 4+15-3*4 =  4+15-12 = 19-12 = 7.\\
\\
For reference, here is a grammar which implements basic mathematical
expressions but is ambiguous:\\
\[ S \rightarrow S+S | S-S | S*S | S/S | n \]
Which could generate expressions such as:\\
n+n*n\\
n-n+n/n-n*n\\
Where n would stand for some number (though each n may stand for a different
number in this example).\\
}
\item \textbf{Next, make the "n" a non-terminal and allow it to produce any
  non-negative integer, including 0 but not numbers with unnecessary leading
0s. Again, make sure it's unambiguous.}
  \item \textbf{Finally, add in the ability for the grammar to produce balanced
    parentheses around terms in a way such that anything inside parentheses
    is evaluated first, in the way that parentheses are normally handled.}
  \item \textbf{Create a pushdown automaton (PDA) that recognizes the language
    generated by the grammar from part c).}

\end{enumerate}
\newpage
\section*{Problem 2} \textbf{Create a PDA that accepts the language $a^ib^jc^k$,
  where $i>j$ and $i<j+k$, in two ways: first, create the PDA without first creating a CFG.
Then, create a CFG and convert that to a PDA.}
\newpage
\section*{Problem 3} \textbf{Create a PDA that accepts the language of all
  words with more a's than b's. Don't go through a CFG, use the stack to do the
  work.}
\newpage
\section*{Problem 4} \textbf{Prove that CFLs are closed under union.}

\end{document}
