%%%%%%%%%%%%%%%%%%%%%%%%%%%%%%%%%%%%%%%%%%%%%%%%%%%%%%%%%%%%%%%%%%%%%
% Use the koma-script document style
\documentclass{scrbook}
%\KOMAoptions{twoside=false} % disable two-side formatting for scrbook
% alternatively, for shorter essay, use the following
% \documentclass{scrartcl}
%%%%%%%%%%%%%%%%%%%%%%%%%%%%%%%%%%%%%%%%%%%%%%%%%%%%%%%%%%%%%%%%%%%%%

%%%%%%%%%%%%%%%%%%%%%%%%%%%%%%%%%%%%%%%%%%%%%%%%%%%%%%%%%%%%%%%%%%%%%
% Useful packages
\usepackage{mathtools}
\usepackage{amssymb,bm,bbold}
\usepackage{enumerate}

\usepackage{hhline}
\usepackage{float}

% CSCI-265
\usepackage{tikz}
\usetikzlibrary{automata, positioning, arrows, arrows.meta}


%=================================
% pre-defined theorem environments
\usepackage{amsthm}
\newtheorem{theorem}{Theorem}
\newtheorem{lemma}{Lemma}
\newtheorem{proposition}{Proposition}
\newtheorem{corollary}{Corollary}
\newtheorem{definition}{Definition}
\newtheorem*{remark}{Remark}
\newtheorem*{assumption}{Assumption}

%=================================
% useful commands
\DeclareMathOperator*{\argmin}{arg\,min}
\DeclareMathOperator*{\argmax}{arg\,max}
\DeclareMathOperator*{\supp}{supp}

\def\vec#1{{\ensuremath{\bm{{#1}}}}}
\def\mat#1{\vec{#1}}

%=================================
% convenient notations
\newcommand{\XX}{\mathbb{X}}
\newcommand{\RR}{\mathbb{R}}
\newcommand{\NN}{\mathbb{N}}
\newcommand{\QQ}{\mathbb{Q}}
\newcommand{\ZZ}{\mathbb{Z}}
\newcommand{\EE}{\mathbb{E}}
\newcommand{\PP}{\mathbb{P}}

\newcommand{\sL}{\mathcal{L}}
\newcommand{\sX}{\mathcal{X}}
\newcommand{\sY}{\mathcal{Y}}

\newcommand{\ind}{\mathbb{1}}

\newcommand{\kleene}{{}^\ast}

%%%%%%%%%%%%%%%%%%%%%%%%%%%%%%%%%%%%%%%%%%%%%%%%%%%%%%%%%%%%%%%%%%%%%
% Typography, change document font
\usepackage[tt=false, type1=true]{libertine}
\usepackage[varqu]{zi4}
\usepackage[libertine]{newtxmath}
\usepackage[T1]{fontenc}

\usepackage[protrusion=true,expansion=true]{microtype}

\author{Guy Matz}

\begin{document}
	
\tikzset{
	->, % makes the edges directed 
%		>='stealth', % makes the arrow heads bold 
	node distance=3cm, % specifies the minimum distance between two nodes. Change if necessary. 
	every state/.style={thick, fill=gray!10}, % sets the properties for each ’state’ node 
	initial text=$ $, % sets the text that appears on the start arrow 
}
	
\title{Title}
% \maketitle

% \tableofcontents
% 
% %\bibliography{bibfile}
% 
% \end{document}

	
\begin{enumerate}
	
\item \textbf{Construct a regular expression defining each of the following languages over the alphabet $\Sigma = \{a, b\}$}

\begin{enumerate}
	\item All words in which a appears tripled, if at all. This means that every clump of a's contains 3 or 6 or 9 or 12, etc. a's.
	$$b^*(aaa)^*b^*$$
    \item All words that contain at least one of the strings S1, S2, S3, or S4.
    
    \item All words that contain exactly 2 b's or exactly 3 b's.
    $$(a^*ba^*ba^*) + (a^*ba^*ba^*ba^*)$$
    \item All strings that end in a double letter. A double letter means the same letter repeated twice - i.e., aa or bb.
    $$(a^*b^*)^*(aa + bb)$$
    \item All strings that do not end in a double letter.
    $$(a^*b^*)^*(ab + ba)$$
    \item All strings that have exactly 1 occurrence of a double letter in them.
    $$((ab)^*(aa)(ba)^*) + ((ba)^*(bb)(ab)^*)$$
    \item All strings in which  the letter b is never tripled. This means no word contains the substring bbb.
    $$ a^*(((b + bb)a)^*a^*) ^*(b + bb) $$
    \item All words in which a is tripled or b is tripled, but not both.
    $$ (b^*(aaa)^*b^*) + (a^*(bbb)^*a^*)  $$
\end{enumerate}
		

\newpage
\item \textbf{If the only difference between $L$ and $L*$ is the word $\epsilon$, is the only difference between $L^2$ and $L*$ the word $\epsilon$? Prove or disprove.}
Note: $ L^2$ is the language consisting of words that are any word in $L$ concatenated with any word in $L$. $L*$ is the language of zero or more words from $L$ all concatenated (in any order, and any word may be used any number of times).
\\\\
Yes.  $L \subseteq L^*$ and $L \subseteq L^2$  so $L^2 \subseteq L^*$ 

\newpage
\item \textbf{Describe, in English, the languages defined by the following regular expressions:}
\begin{enumerate}
  \item $(a+b)*a(\epsilon+bbbb)$
  \item $(a(a+bb)*)*$
  \item $(a(aa)*b(bb)*)*$
  \item $(b(bb)*)*(a(aa)*b(bb)*)*$
  \item $(b(bb)*)*(a(aa)*b(bb)*)*(a(aa)*)*$
  \item $((a+b)a)*$
\end{enumerate}

\newpage
\item \textbf{For each language over $\{a,b\}$, give the state diagram and formal definition of the DFA that recognizes it:}
\begin{enumerate}
  \item The language of all words with exactly 4 letters.
  \item The language of all words that begin or end with a double letter.
  \item The language of all words with only a's or only b's. For clarification, the empty string is not in this language.
\end{enumerate}


\newpage
\item \textbf{One of the defining properties of regular languages is that they're recognized by DFAs.}
\begin{enumerate}
  \item Write "every regular language can be recognized by some DFA" as a first-order formula, and find its complement, contrapositive, and dual. Express each of these as both a first-order formula and a written description in English. Hint: for this example, you can treat "can be recognized by some DFA" as a simple predicate.
  \item What can we infer about nonregular languages from this?
  \item What if we didn't treat "can be recognized by some DFA" as a predicate, but instead wrote it as its own expression in first-order logic?
\end{enumerate}
		
\end{enumerate}

% %\bibliography{bibfile}

\end{document}

