\documentclass[12pt]{scrbook}

%%%%%%%%%%%%%%%%%%%%%%%%%%%%%%%%%%%%%%%%%%%%%%%%%%%%%%%%%%%%%%
% Useful packages
\usepackage{mathtools}
\usepackage{amssymb,bm,bbold}
\usepackage{enumerate}

\usepackage{hhline}
\usepackage{float}

% CSCI-265
\usepackage{tikz}
\usetikzlibrary{automata, positioning, arrows, arrows.meta}


%=================================
% pre-defined theorem environments
\usepackage{amsthm}
\newtheorem{theorem}{Theorem}
\newtheorem{lemma}{Lemma}
\newtheorem{proposition}{Proposition}
\newtheorem{corollary}{Corollary}
\newtheorem{definition}{Definition}
\newtheorem*{remark}{Remark}
\newtheorem*{assumption}{Assumption}

%=================================
% useful commands
\DeclareMathOperator*{\argmin}{arg\,min}
\DeclareMathOperator*{\argmax}{arg\,max}
\DeclareMathOperator*{\supp}{supp}

\def\vec#1{{\ensuremath{\bm{{#1}}}}}
\def\mat#1{\vec{#1}}

%=================================
% convenient notations
\newcommand{\XX}{\mathbb{X}}
\newcommand{\RR}{\mathbb{R}}
\newcommand{\NN}{\mathbb{N}}
\newcommand{\QQ}{\mathbb{Q}}
\newcommand{\ZZ}{\mathbb{Z}}
\newcommand{\EE}{\mathbb{E}}
\newcommand{\PP}{\mathbb{P}}

\newcommand{\sL}{\mathcal{L}}
\newcommand{\sX}{\mathcal{X}}
\newcommand{\sY}{\mathcal{Y}}

\newcommand{\ind}{\mathbb{1}}

\newcommand{\kleene}{{}^\ast}

%%%%%%%%%%%%%%%%%%%%%%%%%%%%%%%%%%%%%%%%%%%%%%%%%%%%%%%%%%%%%%%%%%%%%
% Typography, change document font
\usepackage[tt=false, type1=true]{libertine}
\usepackage[varqu]{zi4}
\usepackage[libertine]{newtxmath}
\usepackage[T1]{fontenc}
%\usepackage[margin=2.5cm]{geometry} % margins
%\usepackage[english]{babel} % language (replace with german or ngerman for german texts)
%\usepackage[utf8]{inputenc} % Umlaute
%\usepackage{amsmath}    % math formulas
%\usepackage{graphicx}    % graphics
%\usepackage{fancyhdr}    % header and footer on every page
%\usepackage{setspace}    % line space (e.g. \singlespacing, \onehalfspacing or \doublespacing)
%\usepackage{enumitem}     % enumerate options
\usepackage[protrusion=true,expansion=true]{microtype}

\author{Guy Matz}

\setlength{\parindent}{0pt} % first line in paragraph will not be indented
%\onehalfspacing        % 1.5 line spacing

\begin{document}

\tikzset{->, % makes the edges directed 
  >=latex, % makes the arrow heads bold 
  node distance=3cm, % specifies the minimum distance between two nodes. Change if necessary. 
  every state/.style={thick, fill=gray!10}, % sets the properties for each ’state’ node 
  initial text=$ $, % sets the text that appears on the start arrow 
}

%% Header on first page with course information etc. %%
  \begin{tabular}{p{15.5cm}}
    {\large \textbf{Computer Theory I}} \\
    Hunter College \\
    Spring 2023  \\
    Instructor: Prof. Tojeira\\
    \hline
  \end{tabular}
  \begin{center}
    {\Large \textbf{Problem Set 10}}
    \vspace{2mm}\\
    Guy Matz
  \end{center}

\section*{Part 1: Implementing a CFG as a PDA:}
\vspace{0.4cm}
\subsection*{Problem 1} Convert the following CFGs to a state diagram of a
PDA\. Do at least one going through CNF and at least one not going through CNF
(hint: one is already in $\mathrm{CNF})$


\begin{enumerate}
  \item $
    \begin{aligned}[t]
        & S \rightarrow A a B \mid S A \\
        & A \rightarrow a A b|B| \epsilon \\
        & B \rightarrow S b \mid b
    \end{aligned}
     $
 \newsavebox{\poopy}
 \sbox{\poopy}{
 	\begin{tabular}{c c c c}
      $\epsilon,S/Aab$ & $\epsilon,A/aAB$ & $\epsilon,B/Sb$ & $a,a/\epsilon$ \tabularnewline
      $\epsilon,S/SA$ & $\epsilon,A/B$ & $\epsilon,B/b$   & $b,b/\epsilon$  \tabularnewline
                      & $\epsilon,A/\epsilon$ & \tabularnewline
  \end{tabular}
  }

  \begin{figure}[H]
  \begin{tikzpicture}
    \node[state, initial] (q1) {$q_1$};
    \node[state, right of=q1] (q2) {$q_2$};
    \node[state, accepting, right of=q2] (q3) {$q_3$};
    \draw
    (q1) edge[above] node{$\epsilon,\epsilon/S$} (q2)
    (q2) edge[loop below] node[align=center] { 
        \usebox{\poopy}
    } (q2)
    (q2) edge[above] node{$\epsilon,Z/\epsilon$} (q3)
    ;
  \end{tikzpicture}
  \end{figure}

   \item $
     \begin{aligned}[t]
       & \mathrm{S} \rightarrow \mathrm{AA}|\mathrm{SB}| \mathrm{b} \\
       & \mathrm{A} \rightarrow \mathrm{AB} \mid \mathrm{a} \\
       & \mathrm{B} \rightarrow \mathrm{SS}|\mathrm{BA}| \mathrm{b}
     \end{aligned}
$\\\\
% \sbox{\poopy}{\begin{tabular}{c c c c}
%     $\epsilon,S/AA$ & $\epsilon,A/AB$ & $\epsilon,B/SS$ & $a,a/\epsilon$ \tabularnewline
%     $\epsilon,S/SB$ & $\epsilon,A/a$ & $\epsilon,B/BA$ & $b,b/\epsilon$  \tabularnewline
%     $\epsilon,S/b$ & & $\epsilon,B/b$ \tabularnewline
%  \end{tabular}
%  }
% IN CNF
ALREADY IN CNF:
 \sbox{\poopy}{\begin{tabular}{c c c}
     $\epsilon,S/AA$ & $\epsilon,A/AB$ & $\epsilon,B/SS$ \tabularnewline
     $\epsilon,S/SB$ & $a,A/\epsilon$ & $\epsilon,B/BA$ \tabularnewline
     $b,S/\epsilon$ & & $b,B/\epsilon$ \tabularnewline
  \end{tabular}
  }

  \begin{figure}[H]
  \begin{tikzpicture}
    \node[state, initial] (q1) {$q_1$};
    \node[state, right of=q1] (q2) {$q_2$};
    \node[state, accepting, right of=q2] (q3) {$q_3$};
    \draw
    (q1) edge[above] node{$\epsilon,\epsilon/S$} (q2)
    (q2) edge[loop below] node[align=center]
    {
      \usebox{\poopy}
    } (q2)
    (q2) edge[above] node{$\epsilon,Z/\epsilon$} (q3)
    ;
  \end{tikzpicture}
  \end{figure}

   \item $
     \begin{aligned}[t]
        & S \rightarrow a S b \mid A B \\
        & A \rightarrow a \mid b \\
        & B \rightarrow B A|S S| b b
     \end{aligned}
     $

 \sbox{\poopy}{\begin{tabular}{c c c c}
     $\epsilon,S/aSb$ & $\epsilon,A/a$ & $\epsilon,B/BA$ & $a,a/\epsilon$ \tabularnewline
     $\epsilon,S/AB$ & $\epsilon,A/b$ &$\epsilon,B/SS$ & $b,b/\epsilon$  \tabularnewline
                     & & $\epsilon,B/bb$ \tabularnewline
  \end{tabular}
  }

  \begin{figure}[H]
  \begin{tikzpicture}
    \node[state, initial] (q1) {$q_1$};
    \node[state, right of=q1] (q2) {$q_2$};
    \node[state, accepting, right of=q2] (q3) {$q_3$};
    \draw
    (q1) edge[above] node{$\epsilon,\epsilon/S$} (q2)
    (q2) edge[loop below] node[align=center]
    {
      \usebox{\poopy}
    } (q2)
    (q2) edge[above] node{$\epsilon,Z/\epsilon$} (q3)
    ;
  \end{tikzpicture}
  \end{figure}
\end{enumerate}

\newpage

\section*{Part 2: Conversions to CNF} Textbook problems: 7.1.1 - 7.1.4 (p.
275): You can find the solutions to the first two problems here:
http://infolab.stanford.edu/ullman/ialcsols/sol7.html

\subsection*{Problem 7.1.1}Find a grammar equivalent to the following, but with
no useless symbols:

$ \begin{aligned}[t]
  S &\rightarrow AB | CA \\
  A &\rightarrow a \\
  B &\rightarrow BC | AB \\
  C &\rightarrow aB | b
  \end{aligned} $

$ \begin{aligned}[t]
  S &\rightarrow AB | CA \\
  A &\rightarrow a \\
  B &\rightarrow BC | AB \\
  C &\rightarrow aB | b
  \end{aligned} $

\newpage \subsection*{Problem 7.1.2}Begin with the following grammar, then
eliminate $\epsilon$-productions, eliminate unit productions, eliminate
useless symbols, then put the grammar into CNF.

$ \begin{aligned}
  S &\rightarrow ASB | \epsilon \\
  A &\rightarrow aAS | a \\
  B &\rightarrow SbS | A | bb
  \end{aligned} 
$
$ \begin{aligned}
  S &\rightarrow ASB | \epsilon \\
  A &\rightarrow aAS | a \\
  B &\rightarrow SbS | A | bb
  \end{aligned} 
$

\subsection*{Problem 7.1.3}As in 7.1.2, convert the following grammar to CNF:

$ \begin{aligned}[t]
  S &\rightarrow 0A0| 1B1| BB \\
  A &\rightarrow C \\
  B &\rightarrow S | A \\
  C &\rightarrow S | \epsilon
\end{aligned}$

$ \begin{aligned}[t]
  S &\rightarrow 0A0| 1B1| BB \\
  A &\rightarrow S \\
  B &\rightarrow S | A \\
\end{aligned}$


\subsection*{Problem 7.1.4}Put this grammar into CNF:

$\begin{aligned}[t]
  S &\rightarrow AAA | B \\
  A &\rightarrow aA | B \\
  B &\rightarrow \epsilon
\end{aligned}$
\\\\\\
$\begin{aligned}[t]
  S &\rightarrow AAA | AA | A \\
  A &\rightarrow aA | a \\
\end{aligned}$


\subsection*{Problem 7.1.x}From a previous exam, put this into CNF:

$\begin{aligned}
 S &\rightarrow BA | bSa \\
 A &\rightarrow B | aB \\
 B &\rightarrow A | \epsilon
\end{aligned}$
\\\\\\
$\begin{aligned}
 S &\rightarrow BA | A | bSa \\
 A &\rightarrow B | aB | a | \epsilon \\
 B &\rightarrow A
\end{aligned}$


\end{document}
